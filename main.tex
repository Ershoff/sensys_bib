\documentclass{llncs}
\usepackage{includes}

\title{Проверка стиля}

\author{Ершов}
\institute{ИППИ}

\graphicspath{{Pics/}}
\begin{document}

\maketitle
\section{Введение}

Преобразование Хафа (ПХ) было предложено Полом Хафом в 1959 году в качестве инструмента анализа фотографий, полученных в пузырьковой камере \cite{hough1959machine}.
Запатентовано же оно было автором в 1962 году \cite{Hough1962patent}.
Согласно патенту, преобразование Хафа -- это такое преобразование изображения, при котором определенным отрезкам на изображении ставятся в соответствие суммы значений пикселей, им принадлежащих.
Аналогом оригинального преобразования Хафа в непрерывном случае является преобразование Радона \cite{van2004short}.

\bibliographystyle{sensys}
\bibliography{biblio}

\end{document}
